\documentclass[a4paper, 12pt]{article}
\usepackage[utf8]{inputenc} 
\usepackage[•]{•}ckage{svg}
\usepackage{hyperref}
\usepackage{tikz}
\usepackage{mathtools}
\usepackage{wrapfig}
\usepackage{enumerate}
\usepackage{amsmath}
\usepackage{adjustbox}
\usepackage{graphicx}
\usepackage[T2A]{fontenc}
\usepackage[utf8]{inputenc}
\usepackage{booktabs,tabularx}
\usepackage{makecell}
\usepackage[most]{tcolorbox}
\usepackage{minted}
\usepackage[T2A]{fontenc} 
\usepackage[english,russian,ukrainian]{babel}


\definecolor{block-gray}{gray}{0.90} % уровень прозрачности (1 - максимум)
\newtcolorbox{test-answer}{colback=block-gray,grow to right by=-10mm,grow to left by=-10mm,
boxrule=1pt,boxsep=0pt,breakable} % настройки области с изменённым фоном
\newtcolorbox{test-answer-long}{colback=block-gray,
boxrule=1pt,boxsep=0pt,breakable} % настройки области с изменённым фоном



\title{Розв’язки задач самостійної роботи}
\author{Бездушний Вадим \textbf{K-24}}
\date{}



\begin{document}

\maketitle
\begin{enumerate}
\item{МНР-обчислюваність функцій $f : N^{n}\rightarrow N.$ \par
\begin{test-answer}
$f : N^{n}\rightarrow N.$ \textit{МНР-обчислювана}, якщо існує МНР-програма, яка
обчислює $f$.
\end{test-answer}
}
\item{Нормальні алгоритми Маркова. Як НА задає вербальне відображення?\par
\begin{test-answer}
Нормальний алгоритм Маркова з алфавітом T - упорядкована послідовність правил вигляду\par
$\alpha \rightarrow \beta $ та 
$\alpha \rightarrow \cdot \beta $, де 
$\alpha, \beta \in T^{*}$ та $\cdot \notin T^{*}$\par
Правила $\alpha \rightarrow \cdot \beta $ - фінальні.\par
Кожен нормальний алгоритм в алфавіті $T$ задає деяке вербальне відображення - $T^{*}\rightarrow T^{*}$
\end{test-answer}
}
\item{Різновиди систем Поста. Нормальні системи Поста. Системи Туе.
\begin{test-answer}
Існують нормальні і антинормальні системи Поста
Правило вигляду $gS\rightarrow Sh$ – правило у нормальній формі.
СП, усі правила якої – у нормальній формі, – нормальна СП\par
\textit{Система Туе} – це СП, усі правила якої мають вигляд $S1\alpha S2 \rightarrow S1 \beta 
S2$, причому $\forall \:S1\alpha S2 \rightarrow S1 \beta S2$ існує симетричне йому $S1\beta S2 \rightarrow S1 \alpha S2$.
\end{test-answer}
\newpage
}
\item{МНР-програма для функції  $f(x, y, z) = x - (y - z).$
\begin{test-answer-long}

$R_0:$ змінна $x$\par
$R_1:$ змінна $y$\par
$R_2:$ змінна $z$\par
$R_3:$ змінна, що прямуватиме до $y-z$\par
Після виконання 6-ої команди $R_2$ прямуватиме до $x - res$
\begin{enumerate}[1.]
\item{$J(1,2,5)$}
\item{$S(2)$}
\item{$S(3)$}
\item{$J(0,0,1)$}
\item{$T(3,1)$}
\item{$Z(2)$}

\item{$J(0,1,12)$}
\item{$S(1)$}
\item{$S(2)$}
\item{$J(0,0,1)$}
\item{$T(2,0)$}


\end{enumerate}
\end{test-answer-long}
\newpage
}
 

\item{МНР-програма для функції  $f(x, y) = max(2x, 2+y).$\par
\begin{test-answer-long}

$R_0:$ змінна $x$\par
$R_1:$ змінна $y+2$\par
$R_2:$ змінна, що прямуватиме до $x$\par
$R_3:$ змінна, що прямуватиме до $2x$
\begin{enumerate}[1.]
\item{$S(1)$}
\item{$S(1)$ Одразу збільшимо $y$ на 2}
\item{$J(0,2,10)$ Кінець роботи алгоритму 2x < y+2, виходимо} 
\item{$J(1,3,9)$ виходимо, якщо маємо вміст $R_3$ рівний $R_1$}
\item{$S(3)$ двічі збільшуємо для 2x}
\item{$J(1,3,9)$}
\item{$S(3)$ двічі збільшуємо для 2x}
\item{$S(2)$}
\item{$J(0,0,3)$}
\item{$T(3,1)$}
\end{enumerate}
% Відповідь знаходиться в $R_1$
\end{test-answer-long}
\newpage
}
\item{Машина Тьюрінга для предиката $x\neq4".$  
\begin{test-answer-long}
Робимо машину Тьюрінга з 5 станами. Переходимо від першого до другого,від другого до третього і так далі. Якщо опинились рівно на 4-ому стані - предикат не виконується( повертаємо $\lambda$).

\begin{enumerate}[]
\item{$q_0| \rightarrow q_1|R$}
\item{$q_1| \rightarrow q_2|R$}
\item{$q_2| \rightarrow q_3|R$}
\item{$q_3| \rightarrow q_4|R$}
\item{$q_4\lambda \rightarrow q^{*}\lambda$

\item{$q_0\lambda \rightarrow q^{*}|$}
\item{$q_1\lambda \rightarrow q^{*}|$}
\item{$q_2\lambda \rightarrow q^{*}|$}
\item{$q_3\lambda \rightarrow q^{*}|$}
\item{$q_4| \rightarrow q^{*}|$}

}


\end{enumerate}

\end{test-answer-long}
\newpage
}
\item{Нормальний алгоритм для функції $f(x, y, z) = (x + y+2) - z.$ }
\begin{test-answer-long}
\begin{enumerate}[]
\item{Спочатку додамо $x$ та $y$: замінимо символ \# між ними та допишемо | на початок}
\item{$q_0| \rightarrow q_1|L$}
\item{$q_1\lambda \rightarrow q_2|R$}
\item{$q_2| \rightarrow q_2|R$}
\item{$q_2\# \rightarrow q_3|L$}
\item{$q_3| \rightarrow q_3|L$}
\item{$q_3\lambda \rightarrow q_4\lambda R$}
\item{Тепер потрібно відняти два числа:}

\item{$q_4|\rightarrow q_5\lambda R$}
\item{$q_5|\rightarrow q_5| R$}
\item{$q_5\#\rightarrow q_5\# R$}
\item{$q_5\lambda \rightarrow q_6\lambda L$}
\item{$q_6|\rightarrow q_7\lambda L$}
\item{$q_7\#\rightarrow q_7\# L$}
\item{$q_7\lambda \rightarrow q_4\lambda R$}
\item{$q_6\#\rightarrow q^{*}|$}
\item{$q_4\#\rightarrow q_8\lambda R$}
\item{$q_8\lambda \rightarrow q^{*}\lambda$}

\end{enumerate}

\end{test-answer-long}
\end{enumerate}
\end{document}
