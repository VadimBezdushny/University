\documentclass[a4paper, 12pt]{article}
\usepackage[utf8]{inputenc} 
\usepackage{svg}
\usepackage{hyperref}
\usepackage{tikz}
\usepackage{mathtools}
\usepackage{wrapfig}
\usepackage{enumerate}
\usepackage{amsmath}
\usepackage{adjustbox}
\usepackage{graphicx}
\usepackage[T2A]{fontenc}
\usepackage[utf8]{inputenc}
\usepackage{booktabs,tabularx}
\usepackage{makecell}
\usepackage[most]{tcolorbox}
\usepackage{minted}
\usepackage[T2A]{fontenc} 
\usepackage[english,russian,ukrainian]{babel}
\DeclarePairedDelimiter{\ceil}{\lceil}{\rceil}

\definecolor{block-gray}{gray}{0.90} % уровень прозрачности (1 - максимум)
\newtcolorbox{test-answer}{colback=block-gray,grow to right by=-10mm,grow to left by=-10mm,
boxrule=1pt,boxsep=0pt,breakable} % настройки области с изменённым фоном
\newtcolorbox{test-answer-long}{colback=block-gray,
boxrule=1pt,boxsep=0pt,breakable} % настройки области с изменённым фоном



\title{Розв’язки задач самостійної роботи}
\author{Бездушний Вадим \textbf{K-24}}
\date{23.02}


\begin{document}

\maketitle
\begin{enumerate}
\item{Операція мінімізації $n$-арних функцій на $N$. \par
\begin{test-answer}
Операція мінімізації $M(n + 1)$-арній функції $g$ зіставляє $n$-арну функцію $f$, яку позначають $M(g)$, що задається співвідношенням

\[f(x_1, \dots, x_n) = \mu_y(g(x_1,...,x_n, y) = 0)\]

Це означає, що для всіх значень $x1, \cdots, x_n$ значення функції $f(x1, \dots, x_n)$ обчислюється так. 

Послідовно обчислюємо значення $g(x_1, \cdots, x_n, y)$ для $y = 0, 1, \dots$

 Перше таке значення $y$, для якого маємо $g(x_1, \dots, x_n, y) = 0$, буде шуканим значенням $f(x_1,\dots, x_n)$. При цьому для всіх $t < y$ значення $g(x_1,\dots, x_n, y)$ мають бути визначені та \ 0.
Операцію M можна розглядати як тотальну функцію із Fn+1 у Fn (при n = 0 – як тотальну функцію із F1 у F1), або як часткову 1-арну функцію на F. 
З визначення зрозуміло, що процес знаходження значення y(g(x1,..., xn, y) = 0) ніколи не закінчиться в таких випадках: 
– значення g(x1,..., xn, 0) невизначене;
– для всіх значень y значення g(x1,..., xn, y) визначене та  0; 
– для всіх t < y значення g(x1,..., xn, t) визначене та  0, а значення g(x1,..., xn, y) невизначене.
Зауваження 1. Для довільного значення x існує єдине значення y = x + 1 таке, що y – (x + 1) = 0. Однак функція y(y – (x + 1) = 0) усюди невизначена, тому що вже 0 – (x + 1) завжди невизначене. Отже, не завжди найменше значення y таке, що g(x1,..., xn, y) = 0, збігається з y(g(x1,..., xn, y) = 0), яке може бути невизначеним, тому що у випадку операції мінімізації таке перше значення y, для якого g(x1,..., xn, y) = 0, знаходиться за допомогою чітко описаного й незалежного від g алгоритму.
З алгоритмічності процесу одержання такого першого y маємо

\end{test-answer}
}
\item{Алгебрa $n$-арних ПРФ. Операторні терми цієї алгебри.\par
\begin{test-answer}
Нормальний алгоритм Маркова з алфавітом T - упорядкована послідовність правил вигляду\par
$\alpha \rightarrow \beta $ та 
$\alpha \rightarrow \cdot \beta $, де 
$\alpha, \beta \in T^{*}$ та $\cdot \notin T^{*}$\par
Правила $\alpha \rightarrow \cdot \beta $ - фінальні.\par
Кожен нормальний алгоритм в алфавіті $T$ задає деяке вербальне відображення - $T^{*}\rightarrow T^{*}$
\end{test-answer}
}
\item{Базові програмовані квазіарні функції на R. ППА програмованих квазіарних функцій на R.
\begin{test-answer}
Існують нормальні і антинормальні системи Поста
Правило вигляду $gS\rightarrow Sh$ – правило у нормальній формі.
СП, усі правила якої – у нормальній формі, – нормальна СП\par
\textit{Система Туе} – це СП, усі правила якої мають вигляд $S1\alpha S2 \rightarrow S1 \beta 
S2$, причому $\forall \:S1\alpha S2 \rightarrow S1 \beta S2$ існує симетричне йому $S1\beta S2 \rightarrow S1 \alpha S2$.
\end{test-answer}
\newpage
}
\item{Дані в мові SIPL, операції на даних.}
\item{Система Поста для функції $f(x, y) = 2x+3y$}
\item{Вкажіть ОТ алгебри КЧРФ для функції $f(x) = \ceil{\log{3}{x}}+2$.}
\item{З‘ясуйте, чи може бути тотальною функція $S^{n+1}(g, g_1 , \dots , g^n )$, якщо $g$ нетотальна.}
\item{Вкажіть ОТ ППА-$ЕQ-N$ для функції $f(x, y, z) = [x/(y+2)]$.}
\end{enumerate}`
\end{document}
