\documentclass[a4paper, 12pt]{article}
\usepackage[utf8]{inputenc} 
\usepackage{svg}
\usepackage{hyperref}
\usepackage{tikz}
\usepackage{mathtools}
\usepackage{wrapfig}
\usepackage{enumerate}
\usepackage{amsmath}
\usepackage{adjustbox}
\usepackage{graphicx}
\usepackage[T2A]{fontenc}
\usepackage[utf8]{inputenc}
\usepackage{booktabs,tabularx}
\usepackage{makecell}
\usepackage[most]{tcolorbox}
\usepackage{minted}
\usepackage[T2A]{fontenc} 
\usepackage[english,russian,ukrainian]{babel}
\DeclarePairedDelimiter{\ceil}{\lceil}{\rceil}

\definecolor{block-gray}{gray}{0.90} % уровень прозрачности (1 - максимум)
\newtcolorbox{test-answer}{colback=block-gray,grow to right by=-10mm,grow to left by=-10mm,
boxrule=1pt,boxsep=0pt,breakable} % настройки области с изменённым фоном
\newtcolorbox{test-answer-long}{colback=block-gray,
boxrule=1pt,boxsep=0pt,breakable} % настройки области с изменённым фоном



\title{Розв’язки задач самостійної роботи}
\author{Бездушний Вадим \textbf{K-24}}
\date{23.02}


\begin{document}

\maketitle
\begin{enumerate}
\item{Операція мінімізації $n$-арних функцій на $N$. \par
\begin{test-answer}
$f : N^{n}\rightarrow N.$ \textit{МНР-обчислювана}, якщо існує МНР-програма, яка
обчислює $f$.
\end{test-answer}
}
\item{Алгебрa $n$-арних ПРФ. Операторні терми цієї алгебри.\par
\begin{test-answer}
Нормальний алгоритм Маркова з алфавітом T - упорядкована послідовність правил вигляду\par
$\alpha \rightarrow \beta $ та 
$\alpha \rightarrow \cdot \beta $, де 
$\alpha, \beta \in T^{*}$ та $\cdot \notin T^{*}$\par
Правила $\alpha \rightarrow \cdot \beta $ - фінальні.\par
Кожен нормальний алгоритм в алфавіті $T$ задає деяке вербальне відображення - $T^{*}\rightarrow T^{*}$
\end{test-answer}
}
\item{Базові програмовані квазіарні функції на R. ППА програмованих квазіарних функцій на R.
\begin{test-answer}
Існують нормальні і антинормальні системи Поста
Правило вигляду $gS\rightarrow Sh$ – правило у нормальній формі.
СП, усі правила якої – у нормальній формі, – нормальна СП\par
\textit{Система Туе} – це СП, усі правила якої мають вигляд $S1\alpha S2 \rightarrow S1 \beta 
S2$, причому $\forall \:S1\alpha S2 \rightarrow S1 \beta S2$ існує симетричне йому $S1\beta S2 \rightarrow S1 \alpha S2$.
\end{test-answer}
\newpage
}
\item{Дані в мові SIPL, операції на даних.}
\item{Система Поста для функції $f(x, y) = 2x+3y$}
\item{Вкажіть ОТ алгебри КЧРФ для функції $f(x) = \ceil{\log{3}{x}}+2$.}
\item{З‘ясуйте, чи може бути тотальною функція $S^{n+1}(g, g_1 , \dots , g^n )$, якщо $g$ нетотальна.}
\item{Вкажіть ОТ ППА-$ЕQ-N$ для функції $f(x, y, z) = [x/(y+2)]$.}
\end{enumerate}`
\end{document}
