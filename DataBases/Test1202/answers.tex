\documentclass[a4paper, 12pt]{article}
\usepackage[utf8]{inputenc} 
\usepackage{svg}
\usepackage{tikz}
\usepackage{mathtools}
\usepackage{wrapfig}
\usepackage{enumerate}
\usepackage{amsmath}
\usepackage{adjustbox}
\usepackage{graphicx}
\usepackage[utf8]{inputenc}
\usepackage{booktabs,tabularx}
\usepackage{makecell}
\usepackage[most]{tcolorbox}
\usepackage{minted}
\usepackage[T2A]{fontenc}      
\usepackage[unicode]{hyperref}
\usepackage[english,russian,ukrainian]{babel}



\definecolor{block-gray}{gray}{0.90} % уровень прозрачности (1 - максимум)
\newtcolorbox{test-answer}{colback=block-gray,grow to right by=-10mm,grow to left by=-10mm,
boxrule=1pt,boxsep=0pt,breakable} % настройки области с изменённым фоном
\newtcolorbox{test-answer-long}{colback=block-gray,
boxrule=1pt,boxsep=0pt,breakable} % настройки области с изменённым фоном



\title{4 запити до БД(12.02)}
\author{Бездушний Вадим \textbf{K-24}}
\date{}

\begin{document}

\maketitle
\begin{enumerate}
\item{Знайти прізвища та організації лекторів, які читають с\textbackslash к (тип) на 3-ому курсі.
\begin{test-answer}
	(П[тип = c\textbackslash к])[Кп] $\rightarrow$ R1\\
	(Г[курс = 3])[Кг] $\rightarrow$ R2\\
	((Р[Кп = Кп]R1)[Кл, Кг])[Кг = Кг]R2)[Кл] $\rightarrow$ R4\\
	(R4[Кл = Кл]Л)[прізв, орг] $\rightarrow$ Res
\end{test-answer}
}
\item{Знайти назви предметів, які Іванчук читає не в аудиторії 205.
\begin{test-answer}
	(Л[прізв = "Іванчук"])[Кл] $\rightarrow$ R1\\
	((Р[Кл = Кл]R1)[ауд $\neq$ 205])[Кп] $\rightarrow$ R2\\
	(П[Кп = Кп]R2)[назва] $\rightarrow$ Res
\end{test-answer}
}

\item{Знайти назви предметів, які Іванчук не читає в аудиторії 205.
\begin{test-answer}
	(Л[прізв = "Іванчук"])[Кл] $\rightarrow$ R1\\
	(Р[Кл = Кл]R1)[Кп] $\rightarrow$ R2\\
	((Р[Кл = Кл]R1)[ауд = 205])[Кп] $\rightarrow$ R3\\
	(П[Кп = Кп](R2[Кп]\textbackslash R3[Кп]))[назва] $\rightarrow$ Res
\end{test-answer}
}

\item{Знайти прізвища лекторів з наук. ступенем доктора, які проводять всі заняття принаймні у ті дні, що і Іванчук. .
\begin{test-answer}
	(Л[прізв = "Іванчук"])[Кл] $\rightarrow$ R1\\
	(Р[Кл = Кл]R1)[день-тижня] $\rightarrow$ R2 //дільник\\
	(Л[Кл = Кл]Р)[Кл, день-тижня]$\rightarrow$ R3 // ділене\\
	((((R3 $\div$ R2)[Кл])[Кл = Кл]Л) $\rightarrow$ R4 \\
	(R4[наук-ступ = "доктор"])[прізв] $\rightarrow$ Res
\end{test-answer}
}

\end{enumerate}

\end{document}
